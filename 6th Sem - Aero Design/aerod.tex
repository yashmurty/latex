\documentclass{beamer}

\title[] % (optional, only for long titles)
{Aero Design}
\subtitle{Final Presentation - Team 10}
\author[] % (optional, for multiple authors)
{Yash Murty \\ Shank The Tank \\ Rohith Grover \\ Ashok Kumar \\ Ravi Prakash}
\institute[IIT Madras] % (optional)
{
  Department of Aerospace Engineering\\
  Indian Institute of Technology\\
  Madras
}
\date[] % (optional)
{}
\subject{Final Presentation}

\usetheme{Warsaw}
\usecolortheme{whale}

\AtBeginSection[]
{
  \begin{frame}
    \frametitle{Table of Contents}
    \tableofcontents[currentsection]
  \end{frame}
}

\begin{document}
	\frame{\titlepage}
	
	\section{Section 1}	
	\begin{frame}
	\frametitle{Some background}
		We start our discussion with some concepts.
	\pause
		The first concept we introduce originates with Erd\H os.
	\end{frame}	
	
	\section{Section 2}
	\begin{frame}
	\frametitle{`Hidden higher-order concepts?'}
	\begin{itemize}[<+->]
	\item The truths of arithmetic which are independent of PA in some 
	sense themselves `{contain} essentially {\color{blue}{hidden higher-order}},
	 or infinitary, concepts'???
	\item `Truths in the language of arithmetic which \ldots
	\item	That suggests stronger version of Isaacson's thesis. 
	\end{itemize}
	\end{frame}
	
	\section{Section 3}
	\begin{frame}{Example of columns 1}
    \begin{columns}[c] % the "c" option specifies center vertical alignment
    \column{.5\textwidth} % column designated by a command
     Contents of the first column
    \column{.5\textwidth}
     Contents split \\ into two lines
    \end{columns}
	\end{frame}
 
	\section{Section 4}
	\begin{frame}{Example of columns 2}
     \begin{columns}[t] % contents are top vertically aligned
     \begin{column}[T]{5cm} % each column can also be its own environment
     Contents of first column \\ split into two lines
     \end{column}
     \begin{column}[T]{5cm} % alternative top-align that's better for graphics
          \includegraphics[height=3cm]{graphic.png}
     \end{column}
     \end{columns}
	\end{frame}
	
	\section{Section 5}
	\begin{frame}
 
   \begin{block}{This is a Block}
      This is important information
   \end{block}
 
   \begin{alertblock}{This is an Alert block}
   This is an important alert
   \end{alertblock}
 
   \begin{exampleblock}{This is an Example block}
   This is an example 
   \end{exampleblock}
 
\end{frame}


	\section{Section 6}
  \begin{frame}
    \frametitle{This is the first slide}
    %Content goes here
	\begin{tabular}{lcccc}
 Class & A & B & C & D \\\hline
 X     & 1 & 2 & 3 & 4 \pause\\
 Y     & 3 & 4 & 5 & 6 \pause\\
 Z     &5&6&7&8
\end{tabular}
		
  \end{frame}

	\section{Section 7}
  \begin{frame}
    \frametitle{This is the second slide}
    \framesubtitle{A bit more information about this}
    %More content goes here

	\begin{columns}
\begin{column}[l]{5cm}
First column
\end{column}
\begin{column}[r]{5cm}
Second column
\end{column}
\end{columns}

  \end{frame}
% etc
\end{document}

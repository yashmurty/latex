\chapter{FUSELAGE DESIGN}
\label{ch3}
This report summarizes the fuselage design of the airplane

\section{FUSELAGE DESIGN}
\label{s:ch3_fuselagedesign}
The fuselage will take a certain amount of load during the flight. We need to design it in the most 
efficient way possible.
\\
There are two ways to go about supporting these loads. 
\begin{itemize}
\item The heavier way is to build the sides from thicker balsa by directly using them as sheets.
\item The lighter way is to build the sides from thinner balsa and reinforce the inside with vertical supports and, to make it more efficient, addition of diagonal bracing between the verticals. Essentially building a truss that's sheeted on the outside.
\end{itemize}
The lightest way to build a fuselage (which also gives the best strength to weight) is to build truss-work sides with gussets at all joints and no sheeting. It requires more work, but results in lighter, stronger and more rigid (for their weight) than any other method in use.

\subsection{Truss Structure}
A truss is a structure that `consists of two-force members only, where the members are organized so that the assemblage as a whole behaves as a single object'. Although this rigorous definition allows the members to have any shape connected in any stable configuration, trusses typically comprise five or more triangular units constructed with straight members whose ends are connected at joints referred to as nodes. In this typical context, external forces and reactions to those forces are considered to act only at the nodes and result in forces in the members which are either tensile or compressive forces.

\subsection{Superior Alternative to Trusses}
A better alternative to the trusses taking the load is to use aluminium rods as the primary load bearing component. The aluminium rods pass through the fuselage. They take all the loads and bending moments from the wings. 
\\
The truss model of the fuselage is present for visual requirements alone so that the RC airplane looks like a conventional scaled down aircraft. 
\\
The Aluminum rod chosen for the wing spar has high margins of safety as seen in Chapter 2. The aluminium spar section is 1cm by 1cm with a thickness of 1mm. As seen, since these Aluminium rods can withstand such high loads and moments generated by the wing, we are using the same aluminium rods for the fuselage construction ( Also, since these are the least dimensions available in the market).

\section{FABRICATION STATUS}
\label{s:ch3_fabricationtillnow}
An update on the fabrication work done until October 1\textsuperscript{st}

\begin{table}[H]
\centering
\makebox[\textwidth]{
\begin{tabular}{|c|c|}
\hline
Stage&Status \\
\hline
Wing Fabrication & Complete \\
Fuselage Fabrication & Complete \\
Landing Gear & Under Fabrication \\
\hline
\end{tabular}}
\caption{Fabrication work done until October 1\textsuperscript{st}}
\label{tab:ch3_fabtillnow}
\end{table}
%
%
%

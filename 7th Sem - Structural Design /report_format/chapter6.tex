\chapter{Flight Testing and challenges}
\label{ch6}

\section{Testing}
The airplane passed all ground tests which included servo tests and motor tests. The initial center of gravity of the plane was behind the wing spar making it tail heavy. The problem was partially corrected by placing a 100g weight at the nose. The result was still a little tail heavy. The problem could not be completely overcome due to our short nose length of fuselage. Hence, the moment arm of any weight was too less to influence the center of gravity unless the weight was more that 500g, which made our plane weight more than 1.5 kg (the rated thrust of the motor). Hence, to avoid any thrust related problems and placing of heavy weights at the nose, the slight tail heaviness was deemed acceptable.
\\
Another minor problem faced was the strength of the servos. Though they worked, it was a possibility that the strain on them could render them unfit during flight. This issue was not served as anything major and hence was pushed aside.
\\
\\
The first flight test was unsuccessful. The aircraft took off the ground successfully, but went into an unrecoverable right turn after which it crashed head on to the ground damaging the wing spar, motor mount, propeller and the motor itself. 
---------INCLUDE CRASH PHOTOS AND EXPLAIN -----------------------



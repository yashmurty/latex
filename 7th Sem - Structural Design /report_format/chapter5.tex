\chapter{ASSEMBLY}
\label{ch5}

The final assembly of the aircraft is described in this chapter

\section{Wing and fuselage}
The Wing is assembled with the fuselage be bolting the wing spar to the fuselage aluminium rod and the main fusleage bulkhead as figure(WHAT) shows. A total of three screws are used to restrict all relative motion between the two structures. Additionally, the fusleage aluminium rod is fastened to the truss by glue and wire to ensure no lateral motions.

\section{Tails and fusleage}
The vertical tail is placed on the aluminium rod using balsa pieces of wood for extra stability. The horizontal tail is also placed on the rod using balsa wood as figure(WHAT) demonstrates

\section{Motor and fuselage}
The motor is bolted to a square plywood frame held to the aluminium rod using L-clamps to dampen out motor vibrations and to ensure the rod is taking the load. The battery is placed right after the motor at the fuselage head with the ESC taped to the outside of the fuselage along with the reciever.

\section{Servos}
Servos in the wing are placed at mid-aileron span within the wing structure secured to the aluminium rod using double-sided tape. The tail servos are fixed to the fuselage to operate the elevator and rudder. All servo wires are connected via extension cables pulled to the reciever placed near the wing spar.

\section{Skin}
The wing is covered with monokote, as are the horizontal and vertical tail. This was done primarily to preserve the airfoil shape which is quite crucial according to our aerodynamic calculations. The fuselage on the other hand is covered with the cheaper coroplast to ensure a more rigid framework (no complex shapes in the fuselage structure).

PUT FIGURE OF FINAL ASSEMBLED AIRCRAFT---------------------------------------------------
